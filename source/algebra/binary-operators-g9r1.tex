% $date: 2017-04-04
% $timetable:
%   g9r1:
%     2017-04-04:
%       3:
%   g9r2:
%     2017-04-05:
%       1:

\worksheet*{Операции на числовых множествах}

% $authors:
% - Александр Савельевич Штерн

Пусть дано некоторое множество чисел $M$.
Оно может состоять из~целых, положительных или совершенно произвольных чисел.
Будем говорить, что на~нём \emph{определена операция,} если есть правило,
которое сопоставляет любым двум элементам этого множества какой-нибудь его
элемент.

\subsection*{Примеры}

\begin{enumerate}

\item
$M = \mathbb{N}$; $x \ast y = x^{y}$.

\item
$M = \mathbb{N}$; $x \ast y = \text{НОД}(x, y)$.

\item
$M = \mathbb{R}$; $x \ast y = x - y$.

\item
$M = \mathbb{R}$; $x \ast y = \lvert x–y \rvert$.

\end{enumerate}
Операции 1 и~3 не~коммутативны и~не~ассоциативны.
Операция 2 коммутативна и~ассоциативна.
Операция 4 коммутативна, но~не~ассоциативна.

\subsection*{Задачи}

\begin{problems}

\item
На~множестве всех вещественных чисел определена антикоммутативная операция.
Докажите, что для любого числа $x$ выполнено условие $x \ast x = 0$.

\item
На~множестве всех вещественных чисел определена ассоциативная антикоммутативная
операция.
Докажите, что для любых чисел $x$, $y$, $z$ выполнено условие
$(x \ast y) \ast z = 0$.

\item
На~множестве всех действительных чисел определена операция, обладающая
следующими свойствами: $x \ast x = 0$, $x \ast (y \ast z) =  (x \ast y) + z$.
Найдите $2017 \ast 1957$.

\item
На~множестве всех целых неотрицательных чисел определена операция $\ast$,
обладающая следующими свойствами:
$0 \ast y = y + 1$,
$(x + 1) \ast 0 = x \ast 1$,
$(x + 1) \ast (y + 1) = x \ast ((x + 1) \ast y)$.
Найдите $3 \ast 2017$.

\item
В~условиях предыдущей задачи найдите $4 \ast 2017$.

\item
На~множестве всех положительных чисел задана операция
$m \circ n = \frac{m + n}{m n + 4}$.
Найдите значение выражения
$(\ldots((2016 \circ 2015) \circ 2014) \circ \ldots \circ 2) \circ 1$.

\item
На~множестве всех действительных чисел определена операция, обладающая
следующим свойством: $(x \ast y) \ast z = x + y + z$.
Докажите, что эта операция есть обычное сложение.

\item
Последовательность натуральных чисел $a_{n}$ построена так, что для любых
$i \neq j$ выполнено свойство $\text{НОД}(a_i, a_j) = \text{НОД}(i, j)$.
Докажите, что при всех $i$ выполнено $a_i = i$.

\item
На~множестве $M = \{ 1, 2, \ldots, n\}$ определено правило, сопоставляющее
любым двум элементом этого множества некоторое целое число так, что для всех
элементов множества выполнено условие $x \ast y + y \ast z + z \ast x = 0$.
Докажите, что можно так подобрать числа $\{ a_{1}, a_{2}, \ldots, a_{n}\}$,
что для любых $i$, $j$ справедливо равенство $i \ast j = a_{i} - a_{j}$.

\end{problems}

