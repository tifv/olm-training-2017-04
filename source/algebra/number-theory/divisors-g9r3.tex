% $date: 2017-04-03
% $timetable:
%   g9r3:
%     2017-04-03:
%       3:

\worksheet*{Число и~его делители}

% $authors:
% - Александр Савельевич Штерн

%Наша задача~--- изучение множества делителей натурального числа с~различных
%точек зрения.
%Совсем просто это множество строится для:
%(а) простого числа $p$: $\{ 1, p \}$;
%(б) числа, которое является степенью простого
%(такие числа называется \emph{примарными}) $p^{\alpha}$:
%$\{ 1, p, p^{2}, p^{3}, \ldots, p^{\alpha}\}$.
%В~остальных случаях изучение может оказаться довольно сложным.
%
%Полезные соображения:
%\begin{enumerate}
%\item
%Делители натурального числа, не~являющегося квадратом, разбиваются на~пары.
%Поэтому натуральное число, не~являющееся квадратом, имеет четное число
%делителей.
%Точные квадраты имеют нечетное число делителей.
%\item
%У~нечетного числа все делители нечетные.
%У~четного числа четных делителей не~меньше, чем нечетных.
%\end{enumerate}
%
%
%\subsection*{Подсчет делителей}
%
%Количество делителей натурального числа можно подсчитать точно.
%Для этого нужно разложить его в~произведение простых множителей.
%
%Натуральное число
%\(
%    n
%=
%    p_{1}^{\alpha_{1}} p_{2}^{\alpha_{2}} \ldots p_{n}^{\alpha^{n}}
%\)
%имеет
%\(
%    d_{n}
%=
%    (\alpha_{1} + 1) (\alpha_{2} + 1) \ldots (\alpha_{n} + 1)
%\)
%различных делителей.
%
%
%\subsection*{Суммирование делителей}
%
%Сумма всех делителей натурального числа
%\(
%    n
%=
%    p_{1}^{\alpha_{1}} p_{2}^{\alpha_{2}} \ldots p_{n}^{\alpha^{n}}
%\)
%вычисляется по~формуле
%\[
%    s(n)
%=
%    (1 + p_{1} + p_{1}^{2} + \ldots + p_{1}^{\alpha_{1}})
%    (1 + p_{2} + p_{2}^{2} + \ldots + p_{2}^{\alpha_{2}})
%    \ldots
%    (1 + p_{n} + p_{n}^{2} + \ldots + p_{n}^{\alpha_{n}})
%\, . \]
%
%Другая запись:
%\(
%    s(p_{1}^{\alpha_{1}} p_{2}^{\alpha_{2}} \ldots p_{n}^{\alpha_{n}})
%=
%    s(p_{1}^{\alpha_{1}}) s(p_{2}^{\alpha_{2}}) \ldots s(p_{n}^{\alpha_{n}})
%\).
%
%Вопрос: у~каких натуральных чисел сумма всех делителей нечетна?
%
%
%\subsection*{Описание четных совершенных чисел}
%
%Натуральное число~$n$ называется \emph{совершенным,} если оно равно сумме
%всех своих делителей кроме самого себя.
%Это условие удобно записывать так $s(n) = 2 n$.
%Самые маленькие совершенные числа~--- $6$ и~$28$ (проверьте!).
%Существуют~ли нечетные совершенные числа~--- неизвестно, а~все четные
%совершенные числа мы сейчас опишем т.\,е. получим для них общую формулу.
%Четное число можно и~очень удобно рассматривать в~виде $2^{n} (2 N + 1)$, где
%$n$ и~$N$~--- произвольные натуральные числа.
%Работая с~такой записью числа, мы не~забудем, что оно четное.
%
%Задачи:
%
%\begin{problems}
%
%\item
%Пусть четное число $2^{n} (2 N + 1)$ совершенно.
%Тогда $(2 N + 1)$ делится на~$2^{n+1} - 1$ без остатка.
%
%\item
%Пусть четное число $2^{n} (2 N + 1)$ совершенно.
%Тогда $(2 N + 1)$ не~имеет других делителей кроме $1$ и~ $2^{n+1} - 1$.
%
%\item
%Четное число $2^{n} (2 N + 1)$ совершенно тогда и~только тогда, когда
%$(2 N + 1) = 2^{n+1} - 1$ и~является простым.
%
%\item
%Найдите еще два совершенных числа.
%
%\end{problems}


%\subsection*{Число и~его делители}

\begin{problems}

\item
Найдите все натуральные числа, у~которых два простых делителя, общее число
делителей равно 6, а~сумма всех делителей равна 28.

\item
Найдите все натуральные числа, имеющие ровно шесть делителей, сумма которых
равна 3500.

\item
Найти все простые числа $p$ такие, что число $p^2 + 11$ имеет ровно шесть
различных делителей.

\item
Петя нашел сумму всех нечетных делителей некоторого четного числа,
а~Вася~--- сумму всех четных делителей этого числа.
Может~ли произведение этих двух чисел быть точным квадратом?

\item
Докажите, что натуральное число $n$ имеет не~более чем различных делителей.

\item
Делитель натурального числа называется \emph{собственным,} если он отличен
от~1 и~самого этого числа.
Натуральное число назовем \emph{восхитительным,} если самый большой
собственный делитель этого числа равен сумме второго по~величине собственного
делителя и~третьего по~величине собственного делителя.
(Например, число~$18$ восхитительное: $9 = 6 + 3$).
Сколько существует восхитительных чисел, не~превосходящих полтора миллиона?

\item
Докажите, что совершенное число не~может быть полным квадратом.

\item
Докажите, что, если совершенное число, большее 6, делится на~3, то~оно
делится на~9.

\item
Докажите, что каждое натуральное число является разностью двух натуральных
чисел, имеющих одинаковое количество простых делителей.

\item
У~натурального числа $N$ выписали в~ряд по~возрастанию все собственные
делители (собственный делитель натурального числа~--- это делитель, отличный
от~$1$ и~самого этого числа).
Оказалось, что в~этом ряду простые и~составные числа чередуются.
Сколько собственных делителей имеет число $N$?

\item
К~натуральному числу прибавили наибольший его собственный делитель и~получили
степень десятки.
Найдите все такие числа.

\item
Натуральное число $N$ называется \emph{хорошим,} если каждый его делитель,
увеличенный на~1, является делителем числа $N + 1$.
Какие числа являются хорошими?
Дайте полный ответ на~этот вопрос.

\end{problems}

