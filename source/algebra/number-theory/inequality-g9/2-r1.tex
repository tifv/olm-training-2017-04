% $date: 2017-04-07
% $timetable:
%   g9r1:
%     2017-04-07:
%       3:

% $matter[no-header,-contained]:
% - verbatim: \worksheet*{Целочисленные неравенства (продолжение)}
% - .[contained]

% $authors:
% - Александр Савельевич Штерн

\claim{Задача о суммах гармонического ряда}
Докажите, что число $1/2 + 1/3 + \ldots + 1/n$ не~может быть целым.


\subsection*{Упражнения}

\begin{enumerate}

\item
Произведение $n$ последовательных натуральных чисел делится на~$n!$.

\item
Число $n!$ не~делится на~$2^{n}$.

\end{enumerate}


\subsection*{Дополнительные задачи}

\begin{problems}

\item
Докажите неравенство
\(
    \text{НОК}(a, b) \cdot \text{НОК}(b, c) \cdot \text{НОК}(a, c)
\geq
    \bigl( \text{НОК}(a, b, c) \bigr)^2
\).

\item
Даны различные натуральные числа $a_1$, $a_2$, \ldots, $a_n$.
Докажите, что НОК всех чисел
\(
    b_{i}
=
    (a_{i} - a_{1}) (a_{i} - a_{2}) \ldots
    (a_{i} - a_{i-1}) (a_{i} - a_{i+1}) \ldots
    (a_{i} - a_{n})
\)
делится на~$(n-1)!$.

\item
Имеется $n$ попарно различных натуральных чисел.
На~доску выписывают их попарные наибольшие общие делители и~попарные наименьшие
общие кратные.
Каким может быть наименьшее количество различных чисел, выписанных на~доске?

\item
Докажите, что при любом натуральном~$n$ можно указать $n$ последовательных
натуральных чисел, среди которых нет ни~одной степени натурального числа
с~показателем степени, большим $1$.

\end{problems}

