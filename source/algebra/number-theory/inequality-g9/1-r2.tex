% $date: 2017-04-06
% $timetable:
%   g9r2:
%     2017-04-06:
%       2:

\worksheet*{Целочисленные неравенства}

% $authors:
% - Александр Савельевич Штерн

\claim{Задача Лиувилля}
Докажите, что ни~при каком простом $p > 5$ число $(p - 1)! + 1$ не~является
степенью числа~$p$.

\subsection*{Задачи для самостоятельного решения}

\begin{problems}

\item
Можно~ли подобрать два различных натуральных числа $x$, $y$ так, чтобы каждое
из~чисел $x^2 + y$, $x + y^2$ было точным квадратом?

\item
Клетки таблицы $200 \times 200$ окрашены в~черный и~белый цвета так, что черных
клеток на~$404$ больше, чем белых.
Докажите, что найдется квадрат $2 \times 2$, в~котором число белых клеток
нечетно.

\item
В~вершинах куба записали восемь различных натуральных чисел, а~на~каждом его
ребре~--- НОД двух чисел, стоящих на~его концах.
Может~ли сумма всех чисел, записанных в~вершинах, быть равна сумме всех чисел,
записанных на~ребрах?

\item
Можно~ли расставить по~кругу 2017 простых чисел так, чтобы квадрат каждого,
уменьшенный на~1, делился на~следующее?

\item
Найдите все натуральные числа $n$, для которых $(n - 1)!$ не делится на $n^2$.

\item
Существует~ли квадратный трехчлен, значение которого при всех целых
значениях аргумента является точной четвертой степенью?

\item
Натуральные числа $a$, $x$ и~$y$, большие 100, таковы, что
$y^2 - 1 = a^2 (x^2 - 1)$.
Какое наименьшее значение может принимать дробь $a / x$?

\item
На~доске написали 100 различных натуральных чисел.
Затем под каждым числом написали сумму этого числа и~наибольшего общего
делителя остальных чисел.
Какое наименьшее количество попарно различных чисел могло при этом получиться?

\item
Существуют~ли три взаимно простых в~совокупности натуральных числа, квадрат
каждого из~которых делится на~сумму двух других?

\item
В~возрастающей бесконечной последовательности натуральных чисел каждое число,
начиная с~$2017$-го, является делителем суммы всех предыдущих чисел.
Докажите, что в~этой последовательности каждое число, начиная с~некоторого
места, равно сумме всех предыдущих чисел.

\end{problems}


