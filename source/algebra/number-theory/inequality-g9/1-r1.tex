% $date: 2017-04-06
% $timetable:
%   g9r1:
%     2017-04-06:
%       1:

\worksheet*{Целочисленные неравенства}

% $authors:
% - Александр Савельевич Штерн

\claim{Задача Лиувилля}
Докажите, что ни~при каком простом $p > 5$ число $(p - 1)! + 1$ не~является
степенью числа~$p$.

\claim{Задача о~размене монет}
В~государстве имеют хождение монеты в~$a$ и~$b$ тугриков, где
$a$ и~$b$ взаимно просты.
Никакие другие монеты хождения не~имеют.
Сколько существует денежных сумм, которые нельзя набрать такими монетами?

\subsection*{Задачи для самостоятельного решения}

\begin{problems}

\item
Клетки таблицы $200 \times 200$ окрашены в~черный и~белый цвета так, что черных
клеток на~$404$ больше, чем белых.
Докажите, что найдется квадрат $2 \times 2$, в~котором число белых клеток
нечетно.

\item
В~вершинах куба записали восемь различных натуральных чисел, а~на~каждом его
ребре~--- НОД двух чисел, стоящих на~его концах.
Может~ли сумма всех чисел, записанных в~вершинах, быть равна сумме всех чисел,
записанных на~ребрах?

\item
Существует~ли квадратный трехчлен, значение которого при всех целых
значениях аргумента является точной четвертой степенью?

\item
Натуральные числа $a$, $x$ и~$y$, большие 100, таковы, что
$y^2 - 1 = a^2 (x^2 - 1)$.
Какое наименьшее значение может принимать дробь $a / x$?

\item
На~доске написали 100 различных натуральных чисел.
Затем под каждым числом написали сумму этого числа и~наибольшего общего
делителя остальных чисел.
Какое наименьшее количество попарно различных чисел могло при этом получиться?

\item
Докажите, что число $1/3 + 1/5 + 1/7 + \ldots + 1/(2n+1)$ не~может быть целым.

\item
На~отрезке натурального ряда ровно $10$ четвертых степеней и~ровно $100$ кубов.
Докажите, что на~этом отрезке не~менее $2000$ точных квадратов.

\item
Даны натуральные числа $x$, $y$, принадлежащие отрезку $[2; 100]$.
Докажите, что среди десяти первых членов последовательности
$a_{n} = x^{2^{n}} + y^{2^{n}}$ найдется составное число.

\item
Существуют~ли три взаимно простых в~совокупности натуральных числа, квадрат
каждого из~которых делится на~сумму двух других?

\item
В~возрастающей бесконечной последовательности натуральных чисел каждое число,
начиная с~$2017$-го, является делителем суммы всех предыдущих чисел.
Докажите, что в~этой последовательности каждое число, начиная с~некоторого
места, равно сумме всех предыдущих чисел.

\end{problems}

