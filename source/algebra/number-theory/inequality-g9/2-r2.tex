% $date: 2017-04-07
% $timetable:
%   g9r2:
%     2017-04-07:
%       1:

% $matter[no-header,-contained]:
% - verbatim: \worksheet*{Целочисленные неравенства (продолжение)}
% - .[contained]

% $authors:
% - Александр Савельевич Штерн

\claim{Задача о~размене монет}
В~государстве имеют хождение монеты в~$a$ и~$b$ тугриков, где
$a$ и~$b$ взаимно просты.
Никакие другие монеты хождения не~имеют.
Сколько существует денежных сумм, которые нельзя набрать такими монетами?


\subsection*{Дополнительные задачи}

\begin{problems}

\item
Докажите, что для любой пары натуральных чисел $m$ и $n$ можно подобрать три
целых числа $a$, $b$, $c$ так, что уравнение $a x + b y = c$ имеет
в~натуральных числах ровно одно решение $x = m$, $y = n$.

\item
Натуральное число~$N$ можно представить в виде суммы квадратов трёх чисел,
каждое из которых делится на $3$, но нельзя представить в виде суммы квадратов
трёх чисел, каждое из которых не делится на $3$.
Докажите, что $N$ делится на 81.

\item
Можно ли подобрать натуральные числа $x$, $y$ так, чтобы выполнялось равенство
$x^2 + x = y^4 + y^3 = y^2 + y$?

\item
Докажите, что при любом натуральном~$n$ можно указать $n$ последовательных
натуральных чисел, среди которых нет ни~одной степени натурального числа
с~показателем степени, большим $1$.

\end{problems}

