% $date: 2017-04-10
% $timetable:
%   g9r2:
%     2017-04-10:
%       1:

\worksheet*{От~квадратных трехчленов к~многочленам высших степеней}

% $authors:
% - Александр Савельевич Штерн

%Квадратный трехчлен (вспоминаем):
%выделение полного квадрата, парабола, ось симметрии, дискриминант
%и~число корней, дискриминант на~графике, теорема Виета.
%
%Кубический многочлен:
%число корней; выделение полного куба, два вида кубической параболы,
%центр симметрии, дискриминант и~число корней, теорема Виета.
%
%Утверждения:
%
%(а) При $p \geq 0$ график многочлена $x^3 + p x + q$ пересекает каждую
%горизонтальную прямую ровно в~одной точке;
%
%(б) при $p < 0$ график пересекает некоторые горизонтальные прямые в~трех
%точках;
%
%(в) при $p < 0$ график имеет один минимум и~один максимум;
%
%(г) абсциссы точек минимума и~максимума противоположны.
%
%Упражнение.
%Три числа $x$, $y$, $z$ подобраны так, что их сумма, произведение и~сумма
%попарных произведений положительны.
%Докажите, что каждое число положительно.

\begingroup
    \def\abs#1{\lvert #1 \rvert}%

\begin{problems}

\item
Существует~ли функция, график которой на~координатной плоскости имеет общую
точку с~любой прямой?

\item
На~доске написано выражение $x^{3} + \ldots x^{2} + \ldots x + \ldots = 0$.
Два школьника по~очереди вписывают вместо многоточий действительные числа.
Цель первого~--- получить уравнение, имеющее ровно один корень.
Может~ли второй ему помешать?

\item
На~координатной плоскости имеется квадрат со~сторонами, параллельными осям
координат.
Этот квадрат поделен на~$64$ равных квадратика прямыми, параллельными осям
координат.
Внутри квадрата движется точка, координаты которой в~каждый момент времени $t$
вычисляются по~формулам
$x = a t^3 + b t^2 + c t + d$, $y = A t^3 + B t^2 + C t + D$.
Докажите, что среди этих $64$~квадратиков найдется такой, внутри которого точка
не~находилась ни~в~какой момент времени.

\item
Целые числа $a$, $b$ и~$c$ таковы, что числа
$a / b + b / c + c / a$ и~$a / c + c / b + b / a$ тоже целые.
Докажите, что $\abs{a} = \abs{b} = \abs{c}$.

\item
Числа $a$, $b$, $c$ таковы, что уравнение $x^3 + a x^2 + b x  + c = 0$ имеет
три корня, и~выполнено условие $a + b + c \in [-2; 0]$.
Докажите, что хотя~бы один из~корней принадлежит отрезку $[0; 2]$.
% Всерос-2008 9.2

\item
Приведённые квадратные трехчлены $f(x)$ и~$g(x)$ таковы, что оба уравнения
$f(g(x)) = 0$ и~$g(f(x)) = 0$ не~имеют корней.
Докажите, что хотя~бы одно из~уравнений $f(f(x)) = 0$ или~$g(g(x)) = 0$ тоже
не~имеет корней.
% Всерос-2007 9.1

\item
Даны два приведённых многочлена: многочлен 4-й степени $P(x)$ и~квадратный
трехчлен $Q(x)$.
Оба многочлена принимают отрицательные значения на~некотором интервале длины
более 2, а~вне этого интервала неотрицательны.
Докажите, что при некотором $x_0$ выполнено неравенство $P(x_0) < Q(x_0)$.
% Всерос-2001 9.2

\end{problems}

\endgroup % \def\abs

