% $date: 2017-04-07
% $timetable:
%   g9r3:
%     2017-04-07:
%       2:

\worksheet*{Квадратный трёхчлен}

% $authors:
% - Александр Савельевич Штерн

%Вспоминаем:
%общий вид; формула корней; дискриминант; график; теорема Виета;
%значения в~замечательных точках.
%
%Упражнение.
%Числа $a$, $b$, $c$ удовлетворяют неравенствам $a + c < b$ и~$c > 0$.
%Докажите неравенство $b^2 \geq 4 a c$.

\subsection*{Некоторые замечания}

\begin{itemize}
\item
Квадратный трехчлен не~может иметь больше двух корней.
\item
Если в~двух парах чисел совпадают суммы и~произведения, то~эти пары совпадают.
\item
Целый корень квадратного трехчлена с~целыми коэффициентами является делителем
свободного члена.
\item
Корни приведённого квадратного трехчлена с~целыми коэффициентами  либо целые
числа, либо иррациональные.
\item
Квадратный трехчлен, у~которого все коэффициенты целые нечетные числа, не~может
иметь целых корней.
\end{itemize}

%\subsection*{Задачи для обсуждения (вводные)}
%
%\item
%Натуральные числа $m$ и~$n$ таковы, что $\lcm(m, n) + \gcd(m, n) = m + n$.
%Докажите, что одно из~чисел $m$ или $n$ делится на~другое.
%
%\item
%Докажите, что два различных квадратных трехчлена с~целыми коэффициентами
%не~могут иметь общий нецелый корень.
%\item
%Приведённый квадратный трехчлен имеет корни х1, х2.
%Найдите его дискриминант.

\subsection*{Задачи для самостоятельного решения}

\begin{problems}

\item
Докажите, что при любых отличных от~нуля числах $a$, $b$, $c$ имеет корень
хотя~бы одно из~квадратных уравнений\enspace
$a x^2 + 2 b x + c = 0$,\enspace
$b x^2 + 2 c x + a = 0$,\enspace
$c x^2 + 2 a x + b = 0$.

\item
Решите систему уравнений\enspace
$x^3 + y^2 = 2$,\enspace
$x^2 + x y + y^2 - y = 0$.

\item
Для действительных чисел $a$, $b$, $c$ выполнены следующие неравенства:
\[
    (a - b + c) (4 a - 2 b + c) < 0
\, , \qquad
    c (a - b + c) < 0
\, . \]
Докажите, что для этих чисел выполнено неравенство
$(a + b + c) (4 a + 2 b + c) \geq 0$.

\item
Найдите все решения уравнения
\[
    1 + \cfrac{1}{1 + \cfrac{1}{1 +
        \raisebox{-1.4ex}{$\raisebox{0.7ex}{$\ddots$} + \cfrac{1}{x}$}}}
=
    x
\]
(в~правой части дробная черта встречается $2017$ раз).

\item
Найдите все целые значения $a$, при которых уравнение $x^2 + a x + a = 0$ имеет
целый корень.

\item
Рассмотрим графики функций $y = x^2 + p x + q$, которые пересекают оси
координат в~трех различных точках.
Докажите, что все окружности, описанные около треугольников с~вершинами в~этих
точках, имеют общую точку.

\item
Длины сторон многоугольника равны $a_{1}$, $a_{2}$, \ldots, $a_{n}$.
Квадратный трехчлен $f(x)$ таков, что $f(a_{1}) = f(a_{2} + \ldots + a_{n})$.
Докажите, что если $A$~--- сумма длин нескольких сторон многоугольника,
$B$~--- сумма длин остальных его сторон, то~$f(A) = f(B)$.

\item
На~доске написано девять приведённых квадратных трехчленов.
Известно, что взятые по~порядку коэффициенты при $x$ образуют арифметическую
прогрессию, и~взятые по~порядку свободные члены тоже образуют арифметическую
прогрессию.
Известно также, что сумма всех трехчленов имеет корень.
Какое наибольшее количество исходных трехчленов может не~иметь корней?
% Всерос-2010, 11-2, 10-2

\item
Приведённый квадратный трехчлен $P(x)$ имеет общий корень с~многочленом
$P(P(P(x)))$.
Докажите, что либо $0$, либо $1$ является корнем этого трехчлена.
% Всерос-2010, 11-1, 9-1

\item
Докажите, что $a^2 + b^2 + c^2 \geq 14$, если $a + 2 b + 3 c \geq 14$.

\end{problems}

