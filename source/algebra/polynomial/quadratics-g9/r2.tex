% $date: 2017-04-08
% $timetable:
%   g9r2:
%     2017-04-08:
%       2:

\worksheet*{Квадратный трёхчлен}

% $authors:
% - Александр Савельевич Штерн

\begingroup
    \def\abs#1{\lvert #1 \rvert}%

%Вспоминаем:
%общий вид; формула корней; дискриминант; график; теорема Виета;
%значения в~замечательных точках.
%
%Упражнение.
%Числа $a$, $b$, $c$ удовлетворяют неравенствам $a + c < b$ и~$c > 0$.
%Докажите неравенство $b^2 \geq 4 a c$.

\subsection*{Некоторые замечания}

\begin{itemize}
\item
Если в~двух парах чисел совпадают суммы и~произведения, то~эти пары совпадают.
\item
Корни приведённого квадратного трехчлена с~целыми коэффициентами  либо целые
числа, либо иррациональные.
\item
Квадратный трехчлен, у~которого все коэффициенты целые нечетные числа, не~может
иметь целых корней.
\item
Если значения квадратного трехчлена в~двух точках совпадают, то~эти точки
расположены симметрично относительно вершины параболы.
\end{itemize}

%\subsection*{Задачи для обсуждения (вводные)}
%
%\item
%Натуральные числа $m$ и~$n$ таковы, что $\lcm(m, n) + \gcd(m, n) = m + n$.
%Докажите, что одно из~чисел $m$ или $n$ делится на~другое.
%
%\item
%Докажите, что два различных квадратных трехчлена с~целыми коэффициентами
%не~могут иметь общий нецелый корень.

\subsection*{Задачи для самостоятельного решения}

\begin{problems}

\item
Корни уравнения $x^2 + p x + q = 0$ являются целыми числами.
Известно, что $p + q = 198$.
Найдите все возможные значения чисел $p$, $q$.

\item
Найдите все положительные значения $a$, при которых оба корня уравнения
$a^2 x^2 + a x + 1 - 7 a^2 = 0$ являются целыми числами.

\item
Квадратный трехчлен $f(x)$ разрешается заменить одним из~трехчленов
$x^2 f(1 / x + 1)$ или $(x - 1)^2 f\bigl(1 / (x-1)\bigr)$.
Можно~ли с~помощью таких операций получить из~трехчлена $x^2 + 4 x + 3$
трехчлен $x^2 + 10 x + 9$?

\item
Решите систему уравнений\enspace
$x^3 + y^2 = 2$,\enspace
$x^2 + x y + y^2 - y = 0$.

\item
Рассмотрим графики функций $y = x^2 + p x + q$, которые пересекают оси
координат в~трех различных точках.
Докажите, что все окружности, описанные около треугольников с~вершинами в~этих
точках, имеют общую точку.

\item
Длины сторон многоугольника равны $a_{1}$, $a_{2}$, \ldots, $a_{n}$.
Квадратный трехчлен $f(x)$ таков, что $f(a_{1}) = f(a_{2} + \ldots + a_{n})$.
Докажите, что если $A$~--- сумма длин нескольких сторон многоугольника,
$B$~--- сумма длин остальных его сторон, то~$f(A) = f(B)$.

\item
Даны два различных приведённых кубических многочлена $F(x)$ и~$G(x)$.
Выписали все корни трех уравнений $F(x) = 0$, $G(x) = 0$ и~$F(x) = G(x)$.
Их оказалось ровно восемь.
Докажите, что самое большое и~самое маленькое из~этих восьми чисел не~могут
одновременно быть корнями многочлена $F(x)$.
% Всерос

\item
Квадратный трехчлен $P(x)$ имеет два различных корня и~при всех $x$
удовлетворяет неравенству $P(x^3 + x) \geq P(x^2 + 1)$.
Найдите сумму корней этого трехчлена.
% Ту-242

\item
Все значения квадратного трехчлена $a x^2 + b x + c$ на~отрезке $[0; 1]$
по~модулю не~превосходят $1$.
Какое наибольшее значение при этом может иметь величина
$\abs{a} + \abs{b} + \abs{c}$?

\item
Три различных числа таковы, что при любой их расстановке на~места коэффициентов
квадратного трехчлена будет получаться трехчлен, имеющий целый корень.
Какие значения может принимать сумма этих чисел?
% Ту-219

\item
Будем говорить, что квадратный трехчлен \emph{переставляет} пару различных
чисел $a, b$, если $f(a) = b$ и~$f(b) = a$.
Может~ли один и~тот~же трехчлен менять местами две различные пары?
% СПб-2014

\item
На~доске написано девять приведённых квадратных трехчленов.
Известно, что взятые по~порядку коэффициенты при $x$ образуют арифметическую
прогрессию, и~взятые по~порядку свободные члены тоже образуют арифметическую
прогрессию.
Известно также, что сумма всех трехчленов имеет корень.
Какое наибольшее количество исходных трехчленов может не~иметь корней?
% Всерос-2010, 11-2, 10-2

\end{problems}

\endgroup % \def\abs

