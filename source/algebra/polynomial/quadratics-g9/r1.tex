% $date: 2017-04-08
% $timetable:
%   g9r1:
%     2017-04-08:
%       1:

\worksheet*{Квадратный трёхчлен}

% $authors:
% - Александр Савельевич Штерн

\begin{problems}

\item
Квадратный трехчлен $f(x)$ разрешается заменить одним из~трехчленов
$x^2 f(1 / x + 1)$ или $(x - 1)^2 f\bigl(1 / (x-1)\bigr)$.
Можно~ли с~помощью таких операций получить из~трехчлена $x^2 + 4 x + 3$
трехчлен $x^2 + 10 x + 9$?

\item
Даны два различных приведённых кубических многочлена $F(x)$ и~$G(x)$.
Выписали все корни трех уравнений $F(x) = 0$, $G(x) = 0$ и~$F(x) = G(x)$.
Их оказалось ровно восемь.
Докажите, что самое большое и~самое маленькое из~этих восьми чисел не~могут
одновременно быть корнями многочлена $F(x)$.
% Всерос

\item
Длины сторон многоугольника равны $a_{1}$, $a_{2}$, \ldots, $a_{n}$.
Квадратный трехчлен $f(x)$ таков, что $f(a_{1}) = f(a_{2} + \ldots + a_{n})$.
Докажите, что если $A$~--- сумма длин нескольких сторон многоугольника,
$B$~--- сумма длин остальных его сторон, то~$f(A) = f(B)$.

\item
Даны квадратные трехчлены $f(x)$, $g(x)$, $h(x)$.
Может~ли уравнение $f(g(h(x))) = 0$ иметь корни $1, 2, 3, 4, 5, 6, 7, 8$?
% ВрМО-531

\item
Квадратный трехчлен $P(x)$ имеет два различных корня и~при всех $x$
удовлетворяет неравенству $P(x^3 + x) \geq P(x^2 + 1)$.
Найдите сумму корней этого трехчлена.
% Ту-242

\item
Будем говорить, что квадратный трехчлен \emph{переставляет} пару различных
чисел $a, b$, если $f(a) = b$ и~$f(b) = a$.
Может~ли один и~тот~же трехчлен менять местами две различные пары?
% СПб-2014

\item
Три различных числа таковы, что при любой их расстановке на~места коэффициентов
квадратного трехчлена будет получаться трехчлен, имеющий целый корень.
Какие значения может принимать сумма этих чисел?
% Ту-219

\item
Даны четыре приведённых квадратных трехчлена.
Сумма любых двух из~них имеет ровно один корень.
Докажите, что среди этих трехчленов не~более двух различных.
% СПб-2005

\item
Дан приведённый квадратный трехчлен $f(x)$.
Известно, что уравнение $f(f(x)) = 0$ имеет четыре различных корня, причем
сумма двух из~них равна $-1$.
Докажите, что свободный член трехчлена не~превосходит $-1/4$.
% ВрМО-800

\item
На~доске написано девять приведённых квадратных трехчленов.
Известно, что взятые по~порядку коэффициенты при $x$ образуют арифметическую
прогрессию, и~взятые по~порядку свободные члены тоже образуют арифметическую
прогрессию.
Известно также, что сумма всех трехчленов имеет корень.
Какое наибольшее количество исходных трехчленов может не~иметь корней?
% Всерос-2010, 11-2, 10-2

\item
На~доске написано $n$~выражений
\( \def\star{\mathord{\ast}}%
    \star x^2 + \star x + \star = 0
\).
Двое по~очереди заменяют одну из~звездочек числом, отличным от~нуля.
Через $3n$ ходов получается $n$ квадратных уравнений.
Первый стремится к~тому, чтобы как можно больше этих уравнений не~имело корней,
а~второй стремится ему помешать.
Какое наибольшее число уравнений без корней может получить первый игрок
независимо от~игры второго?
% ВрМО-488

\end{problems}

