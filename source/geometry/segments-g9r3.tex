% $date: 2017-04-10
% $timetable:
%   g9r3:
%     2017-04-10:
%       3:

\worksheet*{Про отрезки}

% $authors:
% - Фёдор Львович Бахарев

% $build$matter[print]: [[.], [.]]
% $build$matter[print,author]: [[.]]
% $build$style[print,author]:
% - .[resize-to]

\begin{problems}

\item
Через центр вписанной окружности четырехугольника $ABCD$ проведена прямая.
Она пересекает сторону~$AB$ в~точке~$X$ и~сторону~$CD$ в~точке~$Y$;
углы $\angle AXY$ и~$\angle DYX$ равны.
Докажите, что $AX/BX = CY/DY$.

\item
Дан остроугольный треугольник $ABC$.
$B_1$, $C_1$~--- основания высот из~вершин~$B$, $C$ соответственно.
Точка~$D$~--- основание перпендикуляра из~$B_1$ на~$AB$, $E$~--- точка
пересечения перпендикуляра, опущенного из~из~точки~$D$ на~сторону~$BC$,
с~отрезком~$B B_1$.
Докажите, что прямая~$E C_1$ параллельна~$AC$.

\item
Дан биллиард в~форме правильного $1998$-угольника $A_1 A_2 {\ldots} A_{1998}$.
Из~середины стороны $A_1 A_2$ выпустили шар, который, отразившись
последовательно от~сторон $A_2 A_3$, $A_3 A_4$, \ldots, $A_{1998} A_1$
(по~закону <<угол падения равен углу отражения>>), вернулся в~исходную точку.
Докажите, что траектория шара~--- правильный $1998$-угольник.
%Зона 98.9.7.

\item
$AE$ и~$CD$~--- высоты остроугольного треугольника $ABC$.
Биссектриса угла~$B$ пересекает отрезок~$DE$ в~точке~$F$.
На~отрезках~$AE$ и~$CD$ взяли такие точки~$P$ и~$Q$ соответственно, что
четырехугольники $ADFQ$ и~$CEFP$~--- вписанные.
Докажите, что $AP = CQ$.

\item
Даны непересекающиеся окружности~$S_1$ и~$S_2$ и~их общие внешние
касательные~$l_1$ и~$l_2$.
На~$l_1$ между точками касания отметили точку~$A$, а~на~$l_2$~--- точки~$B$
и~$C$ так, что~$AB$ и~$AC$~--- касательные к~$S_1$ и~$S_2$.
Пусть~$O_1$ и~$O_2$~--- центры окружностей~$S_1$ и~$S_2$, а~$K$~--- точка
касания вневписаной окружности треугольника $ABC$ со~стороной~$BC$.
Докажите, что середина отрезка~$O_1 O_2$ равноудалена от~точек~$A$ и~$K$.

\item
Пусть $O$~--- центр описанной окружности остроугольного неравнобедренного
треугольника~$ABC$, точка~$C_1$ симметрична~$C$ относительно~$O$,
$D$~--- середина стороны~$AB$, $K$~--- центр описанной окружности
треугольника~$O D C_1$.
Докажите, что точка~$O$ делит пополам отрезок прямой~$OK$, лежащий внутри
угла~$ACB$.

\end{problems}

