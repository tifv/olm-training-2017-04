% $date: 2017-04-12
% $timetable:
%   g9r1:
%     2017-04-13:
%       2:
%   g9r2:
%     2017-04-13:
%       1:
%   g9r3:
%     2017-04-12:
%       2:

% $build$matter[print,-author]: [[.], [.]]
% $build$style[print,author]: ['.[resize-to]']

\worksheet*{Занятие 2.
Неравенства, связанные с биссектрисами\\ треугольника}

% $authors:
% - Александр Давидович Блинков

\begin{problems}

\item
В~треугольнике $ABC$ биссектриса~$AL$ пересекает описанную окружность
в~точке~$W$, $I$~--- центр вписанной окружности.
Докажите, что:
\\
\subproblem $AI > IL$;
\\
\subproblem $AW + IW > \max(AB, AC)$.

\item
В~треугольнике $ABC$: $I$ и~$I_{a}$~--- центры вписанной и~вневписанной
окружностей соответственно, $R$ и~$r$~--- радиусы описанной и~вписанной
окружностей.
Докажите неравенство $AI \cdot A I_{a} > 4 R r$.

\item
В~треугольнике $ABC$ ($AC > BC$) проведены биссектрисы $AD$ и~$BE$.
Прямая~$DE$ пересекает $AB$ в~точке~$P$.
Докажите, что угол $ACP$~--- тупой.

\item
В~треугольнике $ABC$ проведены биссектрисы $AK$ и~$CM$.
Докажите, что если $AB > BC$, то~$AM > MK > KC$.

\item
Докажите, что:
\[
    \frac{l_{a}}{m_{a}} + \frac{l_{b}}{m_{b}} + \frac{l_{c}}{m_{c}}
>
    1
\]
($l_{k}$~--- биссектрисы треугольника; $m_{k}$~--- его медианы).

\item
Докажите, что
\\
\subproblem
\(
    l_{a}^2 + l_{b}^2 + l_{c}^2
\leq
    p^2
\);
\\
\subproblem
\(
    l_{a} + l_{b} + l_{c}
\leq
    p \sqrt{3}
\)\\
($p$~--- полупериметр треугольника).

\item
Докажите, что в~остроугольном треугольнике
\[
    \frac{1}{l_{a}} + \frac{1}{l_{b}} + \frac{1}{l_{c}}
\leq
    \sqrt{2}
    \left( \frac{1}{a} + \frac{1}{b} + \frac{1}{c} \right)
\]

\item
Докажите, что для любого неравнобедренного треугольника выполняется неравенство
$l_{1}^2 > S \sqrt{3} > l_{2}^2$, где $l_{1}$ и~$l_{2}$~--- наибольшая
и~наименьшая биссектрисы треугольника, $S$~--- его площадь.

\item
Внутри треугольника $ABC$ отмечена точка~$M$.
Прямая~$AM$ вторично пересекает описанную окружность треугольника $ABC$
в~точке~$A_1$.
Докажите, что:
\\[0.5ex]
\subproblem
\( \displaystyle
    \frac{BM \cdot CM}{A_1 M}
\geq
    2 r
\);
\\[0.5ex]
\subproblem
\(
    AM \sin( \angle BMC) + BM \sin( \angle CMA) + CM \sin( \angle AMB)
\leq
    p
\)\\
($r$~--- радиус вписанной окружности, $p$~--- полупериметр треугольника $ABC$).

\end{problems}

