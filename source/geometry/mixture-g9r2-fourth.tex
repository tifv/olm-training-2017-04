% $date: 2017-04-10
% $timetable:
%   g9r2:
%     2017-04-10:
%       2:

\worksheet*{Четвёртые задачи и не только}

% $authors:
% - Фёдор Львович Бахарев

% $build$matter[print]: [[.], [.]]
% $build$matter[print,author]: [[.]]
% $build$style[print,author]:
% - .[resize-to]

\begin{problems}

\item
Трапеция $ABCD$ такова, что на~ее боковых сторонах $AD$ и~$BC$ существуют такие
точки $P$ и~$Q$ соответственно, что
$\angle APB = \angle CPD$, $\angle AQB = \angle CQD$.
Докажите, что точки $P$ и~$Q$ равноудалены от~точки пересечения диагоналей
трапеции.

\item
В~остроугольном неравнобедренном треугольнике $ABC$ биссектриса угла между
высотами $A A_1$ и~$C C_1$ пересекает стороны $AB$ и~$BC$ в~точках $P$ и~$Q$
соответственно.
Биссектриса угла~$B$ пересекает отрезок, соединяющий ортоцентр~$H$
треугольника $ABC$ с~серединой~$M$ стороны~$AC$ в~точке~$R$.
Докажите, что точки $P$, $B$, $Q$ и~$R$ лежат на~одной окружности.

\item
В~остроугольном треугольнике проведены высоты $AA'$ и~$BB'$.
На~дуге $ACB$ описанной окружности треугольника $ABC$ выбрана точка~$D$.
Пусть прямые $AA'$ и~$BD$ пересекаются в~точке~$P$, а~прямые $BB'$ и~$AD$
пересекаются в~точке~$Q$.
Докажите, что прямая~$A'B'$ проходит через середину отрезка~$PQ$.

\item
В~треугольнике $ABC$ проведена биссектриса~$B B_1$.
Перпендикуляр, опущенный из~точки $B_1$ на~$BC$, пересекает дугу~$BC$
описанной окружности треугольника $ABC$ в~точке~$K$.
Перпендикуляр опущенный из~точки~$B$ на~$AK$ пересекает $AC$ в~точке~$L$.
Докажите что точки $K$, $L$ и~середина дуги~$AC$ (не~содержащей точку~$B$)
лежат на~одной прямой.
% Финал 06-07, 4

\item
Пусть $O$~--- центр описанной окружности остроугольного треугольника $ABC$,
$T$~--- центр описанной окружности треугольника $AOC$, $M$~--- середина~$AC$.
На~сторонах $AB$ и~$BC$ выбраны точки $D$ и~$E$ соответственно так, что 
$\angle BDM = \angle BEM = \angle B$.
Докажите, что $BT \perp DE$.
% Финал 03-04, 8

\item
Дан треугольник $ABC$.
Окружность~$\omega$ касается описанной окружности~$\Omega$ треугольника $ABC$
в~точке~$A$, пересекает сторону~$AB$ в~точке~$K$, а~также пересекает
сторону~$BC$.
Касательная~$CL$ к~окружности~$\omega$ такова, что отрезок~$KL$ пересекает
сторону~$BC$ в~точке~$T$.
Докажите, что отрезок~$BT$ равен по~длине касательной, проведенной из~точки~$B$
к~$\omega$.
%Финал 05-06, 4

\end{problems}

