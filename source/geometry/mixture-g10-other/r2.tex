% $date: 2017-04-13
% $timetable:
%   g10r2:
%     2017-04-13:
%       3:
%   g10r3:
%     2017-04-13:
%       2:

\worksheet*{Геометрия}

% $authors:
% - Павел Александрович Кожевников

\begin{problems}

%\item
%Треугольник разбили тремя отрезками, выходящими из~трех его вершин, на~семь
%частей: четыре треугольника и~три четырехугольника.
%Могут~ли все эти три четырехугольника оказаться вписанными?

\item
В~выпуклом шестиугольнике $ABCDEF$:
%$P = AC \cap FB$, $Q = BD \cap AC$, $R = CE \cap BD$,
%$S = DF \cap CE$, $T = EA \cap DF$, $U = FB \cap EA$.
\[
    P = AC \cap FB,\; Q = BD \cap AC,\; R = CE \cap BD,\;
    S = DF \cap CE,\; T = EA \cap DF,\; U = FB \cap EA.
\]
Известно, что $CR = ES$, $BP = FU$, $\angle EAC = \angle BDF = 90^\circ$.
Докажите, что середина~$RS$, середина~$UP$ и~точки $A$ и~$D$ лежат на~одной
окружности или на~одной прямой.

\item
Дан треугольник $ABC$.
Прямая, проходящая через $C$ и~параллельная стороне~$AB$, пересекает
окружность $(ABC)$ в~точке~$C_1$.
Аналогично определяются точки $A_1$ и~$B_1$.
Докажите, что перпендикуляры, опущенные на~стороны $BC$, $CA$, $AB$
из~точек $A_1$, $B_1$, $C_1$ соответственно, пересекаются в~одной точке.

\item
Биссектриса угла~$A$ остроугольного треугольника $ABC$ пересекает сторону~$BC$
в~точке~$L$, а~описанную окружность треугольника~--- в~точке~$N$, отличной
от~$A$.
Точки $K$ и~$M$~--- основания перпендикуляров, опущенных из~$L$
на~стороны $AB$ и~$AC$.
Докажите, что $S_{AKNM} = S_{ABC}$.

%\item
%Продолжения сторон $AB$ и~$CD$ выпуклого четырехугольника $ABCD$ пересекаются
%в~точке $P$, $M$ и~$N$~--- середины диагоналей.
%Докажите, что $S_{PMN} / S_{ABCD} = 1 / 4$.
%% линейность ориентированной площади.
%% на зачете про 33 треугольника

%\item
%Вписанная окружность касается сторон $AB$, $BC$, $CA$
%в~точках $A'$, $B'$, $C'$.
%Докажите, что перпендикуляр из~середины $B'C'$ на~$BC$ и~два аналогичных
%перпендикуляра пересекаются одной точке.
%% Карно

\item
Высоты остроугольного треугольника $ABC$ пересекаются в~точке~$H$.
Окружность, описанная около треугольника $ABH$, пересекает окружность,
построенную на~отрезке~$AC$ как на~диаметре, в~точке~$K$, отличной от~$A$.
Докажите, что прямая~$CK$ делит отрезок~$BH$ пополам.
% Монголия 2014

\item
В~выпуклом четырехугольнике $ABCD$ выполнено:
$\angle B = \angle C = 120^{\circ}$, $AD^2 = AB^2 + BC^2 + CD^2$.
Докажите, что $ABCD$~--- описанный.
% Монголия 2014

\item
В~остроугольном треугольнике $ABC$ сторона~$AB$ меньше стороны~$BC$,
$B H_{b}$~--- высота, точка~$O$~--- центр описанной окружности.
Прямая, проходящая через $H_{b}$ параллельно прямой~$CO$, пересекает
прямую~$BO$ в~точке~$X$.
Докажите, что точка~$X$ и~середины сторон $AB$ и~$AC$ лежат на~одной прямой.
% Кавказ 2017

%\item
%Точки $X$, $Y$ и~$Z$ -- середины высот $AD$, $BE$ и~$CF$ треугольника $ABC$
%соответственно.
%Докажите, что перпендикуляры, опущенные из~$D$ на~$YZ$, из~$E$ на~$ZX$
%и из~$F$ на~$XY$, пересекаются в~одной точке.
%% Европейский математический кубок, 2013
%% Карно…

%\item
%Чевианы $AA'$, $BB'$, $CC'$ треугольника $ABC$ пересекаются в~точке~$P$.
%Пусть $M_{A}$~--- точка пересечения окружностей $BPC'$ и~$CPB'$.
%Аналогично определяются $M_{B}$ и~$M_{C}$.
%Докажите, что $A M_{A}$, $B M_{B}$, $C M_{C}$ пересекаются в~одной точке.

\item
Дан остроугольный треугольник $ABC$.
Перпендикуляр из~$B$ к~прямой~$AC$ пересекает окружность, построенную на~$AC$
как на~диаметре, в~точках $X$ и~$Y$ ($X$ ближе к~$B$, чем $Y$).
Аналогично перпендикуляр из~$C$ к~прямой~$AB$ пересекает окружность,
построенную на~$AB$ как на~диаметре, в~точках $Z$ и~$T$
($Z$ ближе к~$C$, чем $T$).
Докажите, что прямые $XZ$, $YT$ и~$BC$ пересекаются в~одной точке либо
параллельны.
% небось тут и рад. оси пойдут…

%\item
%Дана окружность~$\Omega$ и~точка~$P$ внутри нее.
%Через $P$ проводятся попарно перпендикулярные хорды $AA'$ и~$BB'$.
%Точки $X$, $X'$~--- проекции~$P$ на~прямые $AB$ и~$A'B'$.
%Доказать, что все прямые $XX'$ проходят через одну точку.
%% А. Заславский

%\item
%Треугольник площади~1 разбит двумя чевианами на~4~части.
%Найдите минимум максимальной из~площадей этих 4~частей.
%% аккуратная двигалка

%\item
%Окружности $\omega_1$ и~$\omega_2$ касаются одной прямой в~точках $A$ и~$B$
%соответственно и, кроме того, пересекаются в~точках $X$ и~$Y$, из~которых
%точка~$X$ лежит ближе к~прямой~$AB$.
%Прямая~$AX$ вторично пересекает $\omega_2$ в~точке~$P$.
%Касательная к~$\omega_2$ в~точке~$P$ пересекает прямую~$AB$ в~точке~$Q$.
%Докажите, что $\angle XYB = \angle BYQ$.
%% Польша, 2 этап, 2014
%% проективка… симедиана????

\end{problems}

