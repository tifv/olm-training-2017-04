% $date: 2017-04-03
% $timetable:
%   g9r1:
%     2017-04-04:
%       1:
%   g9r2:
%     2017-04-03:
%       3:
%   g9r3:
%     2017-04-04:
%       2:

\worksheet*{Как решить вторую задачу?}

% $authors:
% - Фёдор Львович Бахарев

\begin{problems}

\item
На~сторонах $AB$, $BC$, $CA$ треугольника $ABC$ выбраны точки $P$, $Q$, $R$
соответственно таким образом, что $AP = CQ$ и~четырехугольник $RPBQ$~---
вписанный.
Касательные к~описанной окружности треугольника $ABC$ в~точках $A$ и~$C$
пересекают прямые $RP$ и~$RQ$ в~точках $X$ и~$Y$ соответственно.
Докажите, что $RX = RY$.
% Финал 05-06, 6
% счет углов и~равенство треугольников! угол между касательной и~хордой.

\item
На~одной стороне угла с~вершиной~$O$ взята точка~$A$, а~на~другой~--- точки $B$
и~$C$, причем точка~$B$ лежит между $O$ и~$C$.
Проведена окружность с~центром~$O_1$, вписанная в~треугольник $OAB$,
и~окружность с~центром~$O_2$, касающаяся стороны~$AC$ и~продолжений
сторон $OA$ и~$OC$ треугольника $AOC$.
Докажите, что если $O_1 A = O_2 A$, то~треугольник $ABC$~--- равнобедренный.
% Финал 01-02, 2
% счет углов, знание стандартных соотношений в~треугольниках.

\item
Вписанная окружность касается сторон $AB$ и~$AC$ треугольника $ABC$
в~точках $X$ и~$Y$ соответственно.
Точка~$K$~--- середина дуги~$AB$ описанной окружности треугольника $ABC$
(не~содержащей точки~$C$).
Оказалось, что прямая~$XY$ делит отрезок~$AK$ пополам.
Чему может быть равен угол $BAC$?
% Финал 07-08, 6
% трезубец

\item
На~стороне~$AB$ треугольника $ABC$ выбрана точка~$D$.
Окружность, описанная около треугольника $BCD$, пересекает сторону~$AC$
в~точке~$M$, а~окружность, описанная около треугольника $ACD$, пересекает
сторону~$BC$ в~точке~$N$ ($M, N \neq C$).
Пусть $O$~--- центр описанной окружности треугольника $CMN$.
Докажите, что прямая~$OD$ перпендикулярна стороне~$AB$.
% Зона 00.9.7.
% Возня со~вписанностями, очень полезно.
% вписанный четрехугольник с~двумя равными сторонами.

\item
Серединный перпендикуляр к~стороне~$AC$ неравнобедренного остроугольного
треугольника $ABC$ пересекает прямые $AB$ и~$BC$ в~точках $B_1$ и~$B_2$
соответственно, а~серединный перпендикуляр к~стороне~$AB$ пересекает
прямые $AC$ и~$BC$ в~точках $C_1$ и~$C_2$ соответственно.
Окружности, описанные около треугольников $B B_1 B_2$ и~$C C_1 C_2$
пересекаются в~точках $P$ и~$Q$.
Докажите, что центр окружности, описанной около треугольника $ABC$, лежит
на~прямой~$PQ$.
% Регион 12-13, 7
% счет углов плюс степень точки

\item
Дан остроугольный треугольник $ABC$.
Окружность, проходящая через вершину~$B$ и~центр $O$ его описанной окружности,
вторично пересекает стороны $BC$ и~$BA$ в~точках $P$ и~$Q$ соответственно.
Докажите, что ортоцентр треугольника $POQ$ лежит на~прямой~$AC$.
% Финал 10-11, 2
% критерий ортоцентра, счет углов, обратить задачу.

\item
Трапеция $ABCD$ с~основаниями $AB$ и~$CD$ вписана в~окружность~$\Omega$.
Окружность~$\omega$ проходит через точки $C$, $D$ и~пересекает
отрезки $CA$, $CB$ в~точках $A_1$, $B_1$ соответственно.
Точки $A_2$ и~$B_2$ симметричны точкам $A_1$ и~$B_1$ относительно середин
отрезков $CA$ и~$CB$ соответственно.
Докажите, что точки $A$, $B$, $A_2$ и~$B_2$ лежат на~одной окружности.
% Финал 13-14, 6
% степень точки, в~обратную сторону

\item
Четырехугольник $ABCD$ описан около окружности.
Биссектрисы внешних углов $A$ и~$B$ пересекаются в~точке~$K$,
внешних углов $B$ и~$C$~--- в~точке~$L$, внешних углов $C$ и~$D$~---
в~точке~$M$, внешних углов $D$ и~$A$~--- в~точке~$N$.
Пусть $K_1$, $L_1$, $M_1$, $N_1$~--- точки пересечения высот
треугольников $ABK$, $BCL$, $CDM$, $DAN$ соответственно.
Докажите, что четырехугольник $K_1 L_1 M_1 N_1$~--- параллелограмм.
% Финал 03-04, 2
% анализ части картинки, параллельность

\item
Точки $A_1$, $B_1$, $C_1$ выбраны на~сторонах $BC$, $CA$ и~$AB$ треугольника
$ABC$ соответственно.
Оказалось, что
\[
    A B_1 - A C_1 = C A_1 - C B_1 = B C_1 - B A_1
\, . \]
Пусть $I_{A}$, $I_{B}$ и~$I_{C}$~--- центры окружностей, вписанных
в~треугольники $A B_1 C_1$, $A_1 B C_1$ и $A_1 B_1 C$, соответственно.
Докажите, что центр окружности, описанной около треугольника $I_A I_B I_C$,
совпадает с~центром окружности, вписанной в~треугольник $ABC$.
% Финал 11-12, 6
% лемма о~трезубце или воробьи.
% нужно найти одно положение для которого верно.

\item
Вписанная окружность треугольника $ABC$ касается сторон $AB$ и~$BC$
в~точках $P$ и~$Q$.
Прямая~$PQ$ пересекает описанную окружность треугольника $ABC$
в~точках $X$ и~$Y$.
Найдите $\angle ABC$, если $\angle XBY = 135^{\circ}$.
% Туй 04.Мл7 апгрейд.
% жесткость

\end{problems}

