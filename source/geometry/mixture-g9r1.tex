% $date: 2017-04-10
% $timetable:
%   g9r1:
%     2017-04-10:
%       1:

\worksheet*{Разнобой}

% $authors:
% - Фёдор Львович Бахарев

% $build$matter[print]: [[.], [.]]
% $build$matter[print,author]: [[.]]
% $build$style[print,author]:
% - .[resize-to]

\begin{problems}

\item
Дана окружность~$\omega$ и~точка~$P$ вне нее.
Проходящая через~$P$ прямая~$\ell$ пересекает $\omega$ в~точках~$A$ и~$B$.
На~отрезке~$AB$ отмечена точка~$C$ такая, что $PA \cdot PB = PC^2$.
Точки $M$ и~$N$~--- середины двух дуг, на~которые хорда~$AB$ разбивает
окружность~$\omega$.
Докажите, что величина $\angle MCN$ не~зависит от~выбора прямой~$\ell$.

\item
Даны непересекающиеся окружности~$S_1$ и~$S_2$ и~их общие внешние
касательные $l_1$ и~$l_2$.
На~$l_1$ между точками касания отметили точку~$A$, а~на~$l_2$~---
точки~$B$ и~$C$ так, что $AB$ и~$AC$~--- касательные к~$S_1$ и~$S_2$.
Пусть $O_1$ и~$O_2$~--- центры окружностей~$S_1$ и~$S_2$, а~$K$~---
точка касания вневписаной окружности треугольника $ABC$ со~стороной~$BC$.
Докажите, что середина отрезка~$O_1 O_2$ равноудалена от~точек $A$ и~$K$.

\item
Медиана~$AM$ треугольника $ABC$ пересекает вписанную в~него окружность
в~точках $X$ и~$Y$.
Известно, что $AB = AC + AM$.
Найдите $\angle XIY$, где $I$~--- центр вписанной окружности.

\item
На~сторонах $BC$ и~$AB$ треугольника $ABC$ стоят точки $X$ и~$Y$ так, что
$\angle BAX = \angle BCY = \alpha$.
Из~вершины~$B$ опущены перпендикуляры $BK$ и~$BL$ на~отрезки $AX$ и~$BY$
соответственно.
Найдите углы треугольника $KLM$, где $M$~--- середина стороны~$AC$.

\item
Через центр вписанной окружности четырехугольника $ABCD$ проведена
прямая.
Она пересекает сторону~$AB$ в~точке~$X$ и~сторону~$CD$ в~точке~$Y$;
углы $\angle AXY$ и~$\angle DYX$ равны.
Докажите, что $AX / BX = CY / DY$.

\item
Стороны треугольника $ABC$ видны из~точки~$T$ под углами $120^\circ$.
Докажите, что прямые, симметричные прямым $AT$, $BT$ и~$CT$ относительно прямых
$BC$, $AC$ и~$AB$ соответственно, пересекаются в~одной точке.

\item
На~дуге~$AC$ описанной окружности треугольника $ABC$ взята произвольная
точка~$P$.
Пусть $I_1$ и~$I_2$~--- центры вписанных окружностей
треугольников $ABP$ и~$CBP$.
Докажите, что описанная окружность треугольника $I_1 I_2 P$ проходит через
некоторую фиксированную точку, не~зависящую от~выбора $P$.

\end{problems}

