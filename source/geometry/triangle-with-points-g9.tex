% $date: 2017-04-11
% $timetable:
%   g9r1:
%     2017-04-12:
%       1:
%   g9r2:
%     2017-04-11:
%       3:
%   g9r3:
%     2017-04-11:
%       1:

% $caption:
%   Ортоцентр, середина стороны, точка пересечания касательных и… ещё одна
%   точка

\worksheet*{%
Ортоцентр, середина стороны, точка пересечения касательных
и{\ldots} еще одна точка!}

% $authors:
% - Александр Давидович Блинков

Пусть $A A_1$ и~$B B_1$~--- высоты остроугольного неравнобедренного
треугольника $ABC$, $H$~--- его ортоцентр, $M$~--- середина~$AB$.
Окружности $\omega$ с~центром~$O$ и~$\omega_1$ с~центром~$O_1$, описанные около
треугольников $ABC$ и~$A_1 B_1 C$ соответственно, вторично пересекаются
в~точке~$P$.

\begin{problems}

\item
Докажите, что точки $M$, $H$ и~$P$ лежат на~одной прямой.

\item
Докажите, что:
\\
\subproblem
окружности, описанные около треугольников $A M A_1$ и~$B M B_1$, проходят через
точку~$P$;
\\
\subproblem
$PM$~--- биссектриса углов $A P A_1$ и~$B P B_1$;
\\
\subproblem
прямая~$PA$ проходит через точку, симметричную точке~$A_1$ относительно
прямой~$CH$.

\item
Пусть $L_1$ и~$L_2$~--- вторые точки пересечения окружности, описанной около
треугольника $A M A_1$, с~прямыми~$BC$ и~$AC$ соответственно,
а~$K_1$ и~$K_2$~--- вторые точки пересечения окружности, описанной около
треугольника $B M B_1$, с~прямыми $AC$ и~$BC$ соответственно.
Докажите, что:
\\
\subproblem $L_1$, $K_1$, $M$ и~$O$ лежат на~одной прямой;
\\
\subproblem $L_2$, $K_2$, $M$ и~$O_1$ лежат на~одной прямой;
\\
\subproblem $L_1$, $K_1$, $L_2$ и~$K_2$ лежат на~одной окружности;
\\
\subproblem прямые $L_1 L_2$, $K_1 K_2$ и~$PM$ пересекаются в~одной точке.

\item
Пусть прямые $A_1 B_1$ и~$AB$ пересекаются в~точке~$S$, $R$~--- середина
отрезка~$CM$.
Докажите, что:
\\
\subproblem точки $C$, $P$ и~$S$ лежат на~одной прямой;
\\
\subproblem прямые $SH$ и~$CM$ перпендикулярны;
\\
\subproblem прямые $OR$ и~$SC$ перпендикулярны.

\item
Пусть касательные к~окружности~$\omega$, проведенные в~точках $A$ и~$B$,
пересекают прямую $A_1 B_1$ в~точках $X$ и~$Y$ соответственно и~пересекаются
в~точке~$Z$.
Докажите, что:
\\
\subproblem
точка~$M$~--- центр вписанной окружности треугольника $XYZ$;
\\
\subproblem
окружности, описанные около треугольников $A M A_1$ и~$B M B_1$, проходят через
точки $X$ и~$Y$ соответственно;
\\
\subproblem
прямые $MH$, $A_1 B_1$ и~$Z C_1$ пересекаются в~одной точке
($C_1$~--- точка пересечения $CH$ и~$AB$).
\\
\subproblem
прямая~$ZP$ проходит через точку $H_{c}$, симметричную $H$ относительно
стороны~$AB$.
\\
\subproblem
описанные окружности треугольников $ABC$ и~$XYZ$ касаются в~точке~$P$.
\\
\subproblem
прямые $AP$, $BC$ и~$Z C_1$ пересекаются в~одной точке.

\end{problems}

