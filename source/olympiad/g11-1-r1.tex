% $date: 2017-04-05
% $timetable:
%   g11r1:
%     2017-04-05:
%       1:
%       2:

\worksheet*{Олимпиада 1}

\begin{problems}

\item
Докажите, что $\sin(\sqrt{x}) < \sqrt{\sin(x)}$ при $0 < x < \pi / 2$.
% ВМО ??.5.1

\item
В~треугольнике $ABC$ проведены медиана~$AM$ и биссектриса~$AL$;
$K$~--- такая точка на~$AM$, что $KL \parallel AC$.
Докажите, что $AL \perp KC$.
% ММО 96.11.3

\item
В~городе Угрюмовске 2\,000\,000 жителей, которые мало общаются друг с~другом.
Тем не~менее, среди любых 2000 жителей найдутся трое попарно знакомых.
Докажите, что в~городе есть четверо попарно знкаомых друг с~другом жителей.
% Питерская городская 2011.9.4 (А. Голованов, С. Берлов)

\item
В~бесконечной последовательности $(x_n)$ первый член $x_1$~--- рациональное
число, большее~$1$, и~$x_{n+1} = x_n + \dfrac{1}{[x_n]}$ при всех
натуральных~$n$.
Докажите, что в~этой последовательности есть целое число.
% ВМО 2007.5.11.4 (А. Голованов)

\itemyx{\thejeolmproblem}{'} \emph{(замена)}
Пусть $M$~--- множество всех положительных рациональных чисел, меньших $1$.
Существует~ли такое подмножество $S$ множества $M$, что любое число из~$M$
представляется в~виде суммы нескольких различных чисел из~$S$ не~менее, чем
одним, и~не~более, чем $1000$ способами?

\end{problems}

